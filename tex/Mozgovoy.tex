\section*{Контрольная работа (A 1 2 d a)}
\subsection*{Задача 1}
Гистограмма:
\begin{figure}[h]
\includegraphics[scale=0.65]{hist.png}
\end{figure}\\
так как $n = 1000$, то $k = \frac{1000^{\frac{1}{3}}}{2} = \frac{10}{2} = 5$\\
Проверим методом $\chi^2$ гипотезу независимости компонент двумерного случайного вектора\\
(для этого используем scipy.stats.chi2\_controgency), получим:
\begin{verbatim}
    p-value: 0.0013793302370199627
\end{verbatim}
То есть p\_value $< 0.05$
\vskip 0.1in \noindent
Проверим гипотезы\\
Спирмен (используя scipy.stats.spearmanr): 
\begin{verbatim}
    coef = -0.03174930822554785
    p-value = 0.3158616269449295
\end{verbatim}
Пирсон (используя scipy.stats.pearsonr): 
\begin{verbatim}
    coef = -0.03618320814722287
    p-value = 0.2529735138640905
\end{verbatim}
\vskip 0.3in



\subsection*{Задача 2}
Проверми КС-методом гипотезу о совпадении законов распределения компонент вектора\\
(используя scipy.stats.ks\_2samp), получим:
\begin{verbatim}
    Ks\_2sampResult(statistic=0.06, pvalue=0.05462666510701526)
\end{verbatim}
Так как pvalue > 0.05, то законы не совпадают, но, что видно из значения, довольно близки
\vskip 0.1in \noindent
Распределение модулей похоже на полунормальное распределение (так как является модулем нормального c $\sigma \approx 0.56$), соответственно плотность функции распределения будет
\begin{gather*}
f(x) =
\begin{cases}
    \frac{\sqrt{2}}{0.56 \sqrt{\pi}} \exp\left(-\frac{x^2}{2 \cdot 0.56^2}\right), \text{ если } x \geqslant 0\\
    0, \text{ если } x \leqslant 0
\end{cases}
\end{gather*}
\vskip 0.4in



\subsection*{Задача 3d}
Произведение плотностей точек $x_1, \ldots, x_n$ это $\frac{1}{\Gamma^n(r)} \cdot (x_1 \cdot \ldots \cdot x_n)^{r-1} \cdot e^{b(x_1 + \ldots + x_n)} \cdot b^{rn}$. Производная по b это $(nr - (x_1 + \ldots + x_n)b)e^{-(x_1 + \ldots + x_n)b} \cdot b^{nr-1}$, ее нули расположены в $0$ и $\frac{rn}{x_1 + \ldots + x_n}$, тогда максимум функции правдоподобия достигается при $b = \frac{rn}{x_1 + \ldots + x_n}$
\vskip 0.4in



\subsection*{Задача 4a}
Смесь плотностей имеет вид $\omega(\alpha) = \int\limits_{\mathbb{R}} v(\alpha, \theta) u(\theta) d \theta$ и $A| \Theta \sim \mathcal{U}(0; \Theta),\ \Theta \sim \mathcal{U}(0; 1)$, тогда
\begin{gather*}
    \int\limits_{\mathbb{R}} v_{A|\Theta}(\alpha, \theta) u_{\Theta}(\theta) d \theta
    = \left(\int\limits_{0}^{\alpha} 0 d \theta + \int\limits_{\alpha}^{1} \frac{1}{\theta} d \theta\right) \cdot 1_{0 < \alpha < 1}
    = -\ln(\alpha) \cdot 1_{0 < \alpha < 1}
\end{gather*}
То есть $w(x) = -\ln(x) \cdot 1_{0 < x < 1}$

